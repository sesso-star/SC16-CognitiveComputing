\documentclass[12pt]{article}
\usepackage[a4paper,margin=1in,footskip=0.25in]{geometry} % set margins
\usepackage[portuguese]{babel}
\usepackage[utf8]{inputenc}
\usepackage{hyperref} 
\usepackage{amsmath}
\usepackage{amssymb}
\usepackage{amsthm}
\usepackage{graphicx}    % needed for include graphics
\usepackage{indentfirst}
\usepackage{float}       % needed for [H] figure placement option
\usepackage{setspace}    % needed for doublespacing
\usepackage{tikz}

% Macros
\renewcommand{\familydefault}{\sfdefault} % sans-serif
\newcommand{\lowtext}[1]{$_{\text{#1}}$}

% Adds ./figures/ to the path of figures
\graphicspath{./figures/}

\title{SC16 Keynote sobre Computação Cognitiva}
\author{Bruno Sesso, Gustavo Estrela de Matos}

\begin{document}
% Espaçamento duplo? Sim até que alguém diga o contrário 
\doublespacing
\begin{titlepage}
    \vfill
    \begin{center}
        \vspace{0.5\textheight}
        \noindent
        Instituto de Matemática e Estatística \\
        Monografia sobre Computação Cognitiva \\
        \vfill
        \noindent
        {\Large SC16 Keynote sobre Computação Cognitiva} \\
        \begin{tabular}{rl}
            {\bf Professor:} & {Alfredo Goldman vel Lejbman} \\
            {\bf Alunos:}    & {Bruno Sesso} \\
                             & {Gustavo Estrela de Matos} \\
        \end{tabular} \\
        \vspace{\fill}
       \bigskip
        São Paulo, \today \\
       \bigskip
    \end{center}
\end{titlepage}

\pagebreak
\tableofcontents
\pagebreak

% o que é computação cognitiva
\section{Introdução}
Neste trabalho, apresentaremos o conceito de computação cognitiva e como
ela é e pode ser usada em sociedade. Para elaborar esse texto, nos 
baseamos principalmente na apresentação de Katharine Frase no Super
Computing 16 (SC16) \href{katherine_frase_keynote}. Katharine trabalha
desde 2015 com a versão voltada a educação do computador IBM Watson, 
famoso por competir (e ganhar) o quiz show Jeopardy em 2011.

\subsection{O que é Computação Cognitiva}
Não há atualmente uma definição amplamente usada na literatura que 
defina computação cognitiva, mas podemos afirmar que um sistema 
é cognitivo se seu funcionamento e, principalmente, sua lógica se
assemelham ao de um cérebro humano. Para que isso aconteça, esses
sistemas podem apresentar características como: funcionamento dependente
de contexto; aprendizado; iteração por meios naturais como fala e visão;
entre outros. Portanto, para se desenvolver um sistema cognitivo é
necessário utilizar conceitos de diversas áreas da computação, como:
aprendizado de máquina, processamento de linguagens naturais,
processamento de sinais, etc.

O Computador IBM Watson é um exemplo de sistema cognitivo, capaz de 
ouvir, interpretar e responder perguntas, o computador ficou famoso 
em 2011 depois de participar e ganhar de humanos em uma edição do jogo 
de respostas e perguntas Jeopardy. Outros exemplos de sistemas 
cognitivos são o Amazon Echo e Google Home, que tem intuito de se tornar
assistentes em residências, capazes de fazer tarefas como reproduzir
músicas e controlar dispositivos inteligentes através de comandos de
voz.

Um fator comum em sistemas cognitivos como Watson é o grande porte 
computacional. O computador da IBM é formado por um cluster de 90 
servidores IBM Power 750 e por 16 terabytes de memória RAM. Já o
Google Home e Amazon Echo rodam em máquinas da nuvem da respectiva
empresa.

Katharina Frase resume um sistema cognitivo em quatro pontos principais:
\begin{itemize}
\item{\textbf {Entendimento}:} o sistema deve ser capaz de entender 
seres humanos de maneira natural, isto é, o usuário deve se comunicar
com o sistema da mesma maneira que se comunica com outros humanos.

\item{\textbf {Raciocínio}:} o sistema deve ser capaz de fazer
implicações lógicas e inferir verdades a partir de dados coletados.

\item{\textbf {Aprendizado}:} o sistema deve ser capaz de usar os dados 
processados para aprender. Por exemplo, se um usuário com uma casa 
inteligente nunca toma café em dias chuvosos, então o sistema deve ser
capaz de inferir que o usuário não tomará café em um dia de chuva.

\item{\textbf {Interação}:} o sistema deve interagir com o usuário de 
forma fácil e natural.
\end{itemize}

\subsection{Vantagens de um Sistema Cognitivo}
O cérebro humano é, apesar de pouco entendido, limitado em alguns 
aspectos que podem ser melhor tratados por uma máquina. A primeira 
limitação humana é não conseguir muitas tarefas ao mesmo tempo. Tente,
por exemplo, fazer duas ou três contas simples ao mesmo tempo; se você
demorou um segundo, saiba que a quantidade de operações que um 
computador como o Watson fez nesse tempo é da ordem de $10^{12}$. Outra
limitação é na memorização de informação: enquanto você precisa escutar
uma música diversas vezes para decorar uma letra, um computador precisa
de apenas uma.

Além dessas limitações mais explícitas do nosso cérebro, existem outras
limitações mais sutis que não costumamos perceber. A mais recorrente 
delas é o viés de confirmação: preferimos consumir informações que são
consistentes com o que já sabemos. Um exemplo desse viés são as 
superstições; se uma pessoa acredita em uma superstição ela costuma
lembrar só de quando a ela funcionou, e esquecer quando ela
não funcionou. Outro exemplo forte do viés de confirmação está na 
insatisfação de times de futebol com árbitros; apesar de maior parte 
dos jogos ocorrerem erros para ambos os lados, os times costumam se
lembrar apenas dos erros que aconteceram contra o seu próprio time,
confirmando a sua ideia de que existe algum favorecimento para o 
adversário.

Um sistema cognitivo pode ser capaz de contornar esses problemas, porque
é capaz de processar uma grande quantidade de dados sem ter nenhum tipo
de viés. Portanto, desde que os dados sejam bem estruturados e com pouco
ruído, um computador cognitivo é capaz de aprender tanto quanto ou até
mais do que um humano. Infelizmente, apesar de gerarmos muitos dados 
atualmente, existem muitas coisas que não geram dado algum ou produz 
dados pouco estruturados e com ruídos.

\section{Arquiteturas de Sistemas Cognitivo}
Como dizemos anteriormente, sistemas cognitivos como o IBM Watson, 
são máquinas de grande porte, caras e que consomem muita energia. Isso
acontece porque os algoritmos que promovem a cognição na máquina são
complexos e dependem do processamento de muitos dados. Porém, como seria
nosso cérebro capaz de realizar atividades parecidas, com melhor 
desempenho e consumo muito menor de energia? Não sabemos, mas isso 
evidencia que nosso modelo computacional pode não ser o melhor para 
criar (simular) a cognição.

Porém, quando tentamos simular o cérebro humano, caímos em um problema
comum em qualquer tipo de simulação: qual nível de detalhes utilizar?
Se usarmos um modelo muito abstrato, longe da realidade do cérebro, é
provável que obtenhamos resultados inconsistentes com a realidade. 
Por outro lado, uma simulação muito detalhada facilmente tornaria-se
computacionalmente intratável, até mesmo para cérebros de animais mais 
simples; o cérebro de um rato, por exemplo, tem por volta de $10^{12}$ 
sinapses, e se usássemos 1 byte para representar cada uma delas, 
precisaríamos de 1 TB de memória.

Diferentes ramos da ciência que estudam o cérebro usam diferentes 
resoluções, porém é bem aceito que o neurônio seja usado como um objeto
básico na maioria dessas áreas. A IBM acredita nesse nível de resolução
e pesando em criar uma máquina mais similar ao cérebro, tanto em sua
estrutura quanto em seu "algoritmo", ela desenvolveu o chip TrueNorth.

O chip TrueNorth implementa em hardware uma rede neural programável, e 
possui 4096 cores, cada um deles simulando 256 neurônios, com 256
sinapses cada um, totalizando por volta de 268 milhões de sinapses em
um chip. O número de transistores nesse chip é de 5,5 bilhões, e na 
época de seu lançamento era o chip com maior número de transistores já
feito. Além disso, o consumo desse chip é de $70$ mW, enquanto
o Watson consome $20$ kW, ou seja, uma diferença da ordem de $10 ^ 6$ W.

A IBM ainda trabalha na criação de um ambiente para programação do 
TrueNorth e garante que o chip é capaz de resolver diversos problemas
de visão, áudio e aprendizado de máquina no geral. Porém, a própria
empresa afirma que o chip ainda não é uma solução para substituição do
modelo de von Neumann, e sim um co-processador. Usando uma analogia com
o próprio cérebro humano, a IBM afirma que a arquitetura clássica pode
ser vista como o lado esquerdo do cérebro, responsável por cálculos,
enquanto o TrueNorth pode ser visto como o lado direito, sensorial e
reconhecedor de padrões.

% Sim. Nos últimos anos conseguimos:
% 1, melhores algoritmos de machine learning
% 2, mais dados
% 3, melhoras em performance
% 4, mais investimentos
\pagebreak
\section{É Possível Construir um Sistema Cognitivo?}
Nos últimos anos vimos um grande avanço tecnológico principalmente na
área de computação. Por conta desse grande avanço passou-se a ser possível
a construção de sistemas computacionais mais avançados, dentre eles a
criação e evolução de sistemas cognitivos. Como vimos anteriormente, 
sistemas cognitivos tentam de alguma forma imitar a forma como os seres
humanos pensam. No entanto esse tipo de computação exige bastante poder
computacional, fazendo com que esse tipo de paradigma não fosse 
implementavel até recentemente.

Vemos que avanços na área de inteligência artificial, como 
machine learning, redes neurais, etc.,  também foram cruciais para
computação computação cognitiva. Vemos nos útlimos anos não somente a
popularização de tais técnicas como também os bons resultados que 
elas proporcionam. Além disso, nunca na história da humanidade houveram
tantos dados como atualmente. Essa quantidade de dados nos permite não
somente usar como dados de treinamento para sistemas cognitivos como
também nos apresenta uma enorme quantidade de informação que pode ser 
relacionada. Como uma quantidade tão grande de informação como essa é
difícil de ser processada por um ser humano, ter esse trabalho feito
por um computador possa ser algo de muito lucro.

Com resultados positivos como tais, consequentemente há um grande
número de investimentos nessas áreas. Todos esses fatores em conjunto 
proporcionam atualmente um ótimo quadro para o avanço em pesquisas e
desenvolvimento de sistemas cognitivos. Nota-se ainda que esses fatores
não existiam até recentemente.

\pagebreak
\section{Uso de Computação Cognitiva}
A grande vantagem em computação cognitiva está no fato desses sistemas
poderem pensar como o ser humano e analisar grandes quantidades de dados.
Descreveremos a seguir dois exemplos:
\subsection{No Ambiente Médico}
Imagine o caso aonde um médico está atendendo um paciente e durante esse
processo ele precisa dar as costas ao paciente para escrever no 
computador. Esse é um cenário não adequado, pois pesquisas indicam que
virar as costas ao paciente gera uma perda de contato que pode ser 
prejudicial para a análise médica. Por conta disso muitos médicos deixam
de usar o computador no seu ambiente de trabalho. Isso também significa
que eles deixam de usufruir da capacidade do computador de analisar, por
 exemplo, os dados de todos os pacientes para inferir alguma informação.
 Esse tipo de comportamento também é verificado em muitas outras áreas
 onde a interação entre o computador e o ser humano não é feita de uma
 forma favorável e portanto o usuário deve adaptar o seu "Workflow" para
 o computador.
 Em um ambiênte onde pudessemos utilizar computação cognitiva, como o
 computador interage naturalmente com o ser humano, ele poderia estar
 na sala do médico durante a conversa com o paciênte e a partir da 
 conversa ele pudesse autmaticamente relacionar dados do banco de dados
 do computador com as informações sendo adquiridas do paciente. Ele
 então poderia sugerir métodos de tratamento favoraveis as preferências
 do paciente. Por exemplo para um tratamento de câncer ele poderia saber
 que o paciênte prefere tratamentos em que não há perda de cabelo e 
 dessa forma sugerir um tratamento mais adequado.

\subsection{No Ambiente Educacional}
Em um ambiente educacional, muito frequentemente, um professor recebe 
ao início de um período letivo uma lista com o nome de todos os seus 
alunos. E normalmente esse é toda a informação que o professor tem sobre
eles. Após somente algumas semanas o professor passa a identificar os
problemas enfrentados por cada aluno e relacionar esses problemas. No
entanto em um ambiênte onde pudesse haver algum sistema cognitivo, o 
computador poderia utilizar as informações que a escola tem sobre cada
aluno (históricos, informações pessoais, etc) para indicar quais alunos 
necessitam de mais atenção, quais aprendem de melhor forma e até mesmo
organizar turmas de alunos de uma forma mais efetiva.

\section{Conclusão}
No apresentação de Katherine Frase na SC16 seu principal objetivo era
em explicar o que o termo Cognitive Computing era. Ficou claro que 
Cognitive Computing é, em termos gerais, criar sistemas computacionais
que sejam semelhantes a forma como o ser humano pensa e além disso
que interaja com os seres humanos de forma natural, por exemplo através
de vóz.

Além disso notamos que na maior parte de nossas pesquisas os resultados
encontrados eram artigos de funcionarios da IBM, páginas do site da IBM 
e na maior parte dos casos com algo relacionado à IBM. A palestrante
Katharine Frase também é funcionária da IBM e o próprio termo Computação
Cognitiva foi criado por um grupo de especialistas de algumas empresas
incluindo a IBM. Essas evidencias nos fazem acreditar que o termo 
Computação Cognitiva é uma estratégia de marketing para colocar uma nova
"Buzzword" em algo que já existe (um conjunto de machine learning, big 
data e boa interação humano-computador). Apesar disso a ideia de criar 
sistemas desse tipo não deixa de ser muito boa e com ótimas expectativas
para o futuro.

\pagebreak
\begin{thebibliography}{1}
\bibitem{katherine_frase_keynote} Video: SC16 Introduction and Keynote
    Katharine Frase. Acessível em: 
    \url{https://www.youtube.com/watch?v=s0KzMK_rh_g&t=1483s}
\end{thebibliography}

\end{document}


