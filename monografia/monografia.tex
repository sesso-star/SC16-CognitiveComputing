\documentclass[12pt]{article}
\usepackage[a4paper,margin=1in,footskip=0.25in]{geometry} % set margins
\usepackage[portuguese]{babel}
\usepackage[utf8]{inputenc}
\usepackage{hyperref} 
\usepackage{amsmath}
\usepackage{amssymb}
\usepackage{amsthm}
\usepackage{graphicx}    % needed for include graphics
\usepackage{indentfirst}
\usepackage{float}       % needed for [H] figure placement option
\usepackage{setspace}    % needed for doublespacing
\usepackage{tikz}

% Macros
\renewcommand{\familydefault}{\sfdefault} % sans-serif
\newcommand{\lowtext}[1]{$_{\text{#1}}$}

% Adds ./figures/ to the path of figures
\graphicspath{./figures/}

\title{SC16 Keynote sobre Computação Cognitiva}
\author{Bruno Sesso, Gustavo Estrela de Matos}

\begin{document}
% Espaçamento duplo? Sim até que alguém diga o contrário 
\doublespacing
\begin{titlepage}
    \vfill
    \begin{center}
        \vspace{0.5\textheight}
        \noindent
        Instituto de Matemática e Estatística \\
        Monografia sobre Computação Cognitiva \\
        \vfill
        \noindent
        {\Large SC16 Keynote sobre Computação Cognitiva} \\
        \begin{tabular}{rl}
            {\bf Professor:} & {Alfredo Goldman vel Lejbman} \\
            {\bf Alunos:}    & {Bruno Sesso} \\
                             & {Gustavo Estrela de Matos} \\
        \end{tabular} \\
        \vspace{\fill}
       \bigskip
        São Paulo, \today \\
       \bigskip
    \end{center}
\end{titlepage}

\pagebreak
\tableofcontents
\pagebreak

% o que é computação cognitiva
\section{Introdução}


% Sim. Nos últimos anos conseguimos:
% 1, melhores algoritmos de machine learning
% 2, mais dados
% 3, melhoras em performance
% 4, mais investimentos
\pagebreak
\section{É Possível Construir um Sistema Cognitivo?}

\pagebreak
\section{Uso de Computação Cognitiva}
\subsection{No Ambiente Médico}
\subsection{No Ambiente Educacional}


\pagebreak
\begin{thebibliography}{1}
\bibitem{katherine_frase_keynote} Video: SC16 Introduction and Keynote
    Katharine Frase. Acessível em: 
    \url{https://www.youtube.com/watch?v=s0KzMK_rh_g&t=1483s}
\end{thebibliography}

\end{document}


