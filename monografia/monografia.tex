\documentclass[12pt]{article}
\usepackage[a4paper,margin=1in,footskip=0.25in]{geometry} % set margins
\usepackage[portuguese]{babel}
\usepackage[utf8]{inputenc}
\usepackage{hyperref} 
\usepackage{amsmath}
\usepackage{amssymb}
\usepackage{amsthm}
\usepackage{graphicx}    % needed for include graphics
\usepackage{indentfirst}
\usepackage{float}       % needed for [H] figure placement option
\usepackage{setspace}    % needed for doublespacing
\usepackage{tikz}

% Macros
\renewcommand{\familydefault}{\sfdefault} % sans-serif
\newcommand{\lowtext}[1]{$_{\text{#1}}$}

% Adds ./figures/ to the path of figures
\graphicspath{./figures/}

\title{SC16 Keynote sobre Computação Cognitiva}
\author{Bruno Sesso, Gustavo Estrela de Matos}

\begin{document}
% Espaçamento duplo? Sim até que alguém diga o contrário 
\doublespacing
\begin{titlepage}
    \vfill
    \begin{center}
        \vspace{0.5\textheight}
        \noindent
        Instituto de Matemática e Estatística \\
        Monografia sobre Computação Cognitiva \\
        \vfill
        \noindent
        {\Large SC16 Keynote sobre Computação Cognitiva} \\
        \begin{tabular}{rl}
            {\bf Professor:} & {Alfredo Goldman vel Lejbman} \\
            {\bf Alunos:}    & {Bruno Sesso} \\
                             & {Gustavo Estrela de Matos} \\
        \end{tabular} \\
        \vspace{\fill}
       \bigskip
        São Paulo, \today \\
       \bigskip
    \end{center}
\end{titlepage}

\pagebreak
\tableofcontents
\pagebreak

% o que é computação cognitiva
\section{Introdução}
Neste trabalho, apresentaremos o conceito de computação cognitiva e como
ela é e pode ser usada em sociedade. Para elaborar esse texto, nos 
baseamos principalmente na apresentação de Katharine Frase no Super
Computing 16 (SC16) \href{katherine_frase_keynote}. Katharine trabalha
desde 2015 com a versão voltada a educação do computador IBM Watson, 
famoso por competir (e ganhar) o quiz show Jeopardy em 2011.

\subsection{O que é Computação Cognitiva}
Não há atualmente uma definição amplamente usada na literatura que 
defina computação cognitiva, mas podemos afirmar que um sistema 
é cognitivo se seu funcionamento e, principalmente, sua lógica se
assemelham ao de um cérebro humano. Para que isso aconteça, esses
sistemas podem apresentar características como: funcionamento dependente
de contexto; aprendizado; iteração por meios naturais como fala e visão;
entre outros. Portanto, para se desenvolver um sistema cognitivo é
necessário utilizar conceitos de diversas áreas da computação, como:
aprendizado de máquina, processamento de linguagens naturais,
processamento de sinais, etc.

O Computador IBM Watson é um exemplo de sistema cognitivo, capaz de 
ouvir, interpretar e responder perguntas, o computador ficou famoso 
em 2011 depois de participar e ganhar de humanos em uma edição do jogo 
de respostas e perguntas Jeopardy. Outros exemplos de sistemas 
cognitivos são o Amazon Echo e Google Home, que tem intuito de se tornar
assistentes em residências, capazes de fazer tarefas como reproduzir
músicas e controlar dispositivos inteligentes através de comandos de
voz.

Um fator comum em sistemas cognitivos como Watson é o grande porte 
computacional. O computador da IBM é formado por um cluster de 90 
servidores IBM Power 750 e por 16 terabytes de memória RAM. Já o
Google Home e Amazon Echo rodam em máquinas da nuvem da respectiva
empresa.

Katharina Frase resume um sistema cognitivo em quatro pontos principais:
\begin{itemize}
\item{\textbf {Entendimento}:} o sistema deve ser capaz de entender 
seres humanos de maneira natural, isto é, o usuário deve se comunicar
com o sistema da mesma maneira que se comunica com outros humanos.

\item{\textbf {Raciocínio}:} o sistema deve ser capaz de fazer
implicações lógicas e inferir verdades a partir de dados coletados.

\item{\textbf {Aprendizado}:} o sistema deve ser capaz de usar os dados 
processados para aprender. Por exemplo, se um usuário com uma casa 
inteligente nunca toma café em dias chuvosos, então o sistema deve ser
capaz de inferir que o usuário não tomará café em um dia de chuva.

\item{\textbf {Interação}:} o sistema deve interagir com o usuário de 
forma fácil e natural.

\end{itemize}

% Sim. Nos últimos anos conseguimos:
% 1, melhores algoritmos de machine learning
% 2, mais dados
% 3, melhoras em performance
% 4, mais investimentos
\pagebreak
\section{É Possível Construir um Sistema Cognitivo?}

\pagebreak
\section{Uso de Computação Cognitiva}
\subsection{No Ambiente Médico}
\subsection{No Ambiente Educacional}


\pagebreak
\begin{thebibliography}{1}
\bibitem{katherine_frase_keynote} Video: SC16 Introduction and Keynote
    Katharine Frase. Acessível em: 
    \url{https://www.youtube.com/watch?v=s0KzMK_rh_g&t=1483s}
\end{thebibliography}

\end{document}


